\section{Problems, issues and points learned}
% In addition to listing problems and issues with the document or with the implementation process, this is also a spot to reflect upon the project and discuss what you would have done differently if you were to start again from scratch.

\paragraph{When one starts} to develop a game, one can also assume there will be some issues. There may be technical issues encountered in our code that needs to be sorted out, things that takes a lot of time to learn, issues in the group, with the reports to be written or other things that affect the project.

\paragraph{The first problem} was with libgdx, that none of us had any experience with it nor with Android except the introductory exercise. One thing is to know how things may be done, another thing is to know if the way you're doing it is the "recommended" way. Without any experience this is almost impossible to know.

\paragraph{This leads us to} another problem, which is that when the whole group has a limited grasp of the technology at hand, it's hard to go through with code reviews, or even work on code others have written. We had a few disagreements about where code should go, or how code worked. As one of the main QA attributes of the project is modifiability, we also had these discussions so that we'd be sure that it made sense to place the code where we did, so any other new developers had an easy job of maneuvering the project and the code base. Another small issue that relates to this was inexperience with git and GitHub among some of the groups members. It took some time for this to be discovered and to be dealt with, but when we finally did it made the development go faster and we communicated a lot better. 


\paragraph{Another issue} we've encountered is that it's hard to agree on a procedure surrounding a project. It may be easy to establish a routine, but it's hard to enforce it throughout the lifetime of the project.


\paragraph{We've also learned} a great deal of things. First and foremost we've experienced how things seem much easier when you're planning them, compared to when you're actually doing them. This is also shows how important it is to plan out the architecture and how the code should be structured in advance, because when you're a part of a development team, one cannot just do as one wants all the time. Communication is key to a great product, and in this case we did communicate as well as we could all the time. We also experienced how having great tools, like Trello, Slack and GitHub, is worth nothing if the team does not know how to use it, doesn't want to or simply forgets to do so.

\paragraph{Of course we} also learned a great deal about Android, libgdx, NodeJS and PostgreSQL, the technologies we used for development. It's one thing to read about these things, but a completely other thing to actually use it to implement something.
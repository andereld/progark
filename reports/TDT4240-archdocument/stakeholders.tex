\section{Stakeholders}

As we do not plan to release the game, we have not taken concerns like publishing or profits, or stakeholders like publishers (Google Play for Android or App Store for iOS) into consideration.

\begin{center}
\begin{longtable}{|p{3cm} | p{10cm}|}
    \hline
        \label{tab:stakeholders}
        \textbf{Stakeholder} & \textbf{Concern} \\ \hline \hline
        User & 
        \textbf{Consistency with requirements:} The game should act and feel like it is supposed to, according to the functional requirements. 
        \newline \textbf{Performance:} It must perform in such a way that it isn't a nuisance for the user. 
        \newline \textbf{Reliability:} Should crash or stall as little as possible. 
        \newline \textbf{Usability:} Must be easy and intuitive to use.
        \newline \textbf{Playability:} Must be fun to play.\\ \hline
        Project\newline Manager &
        \textbf{Schedule estimation:} The game is to be finished in a short amount of time, and we have definite deadlines to take into consideration.
        \newline \textbf{Progress tracking:} It is important to know how far the project has come at any given time.
        \newline \textbf{Requirements traceability:} One should be able to easily assess if a given requirement is being met. 
        \newline \textbf{Effectiveness:} The team must be working effectively together and not wasting time.
        \newline \textbf{Communication:} It is vital to maintain good communication within the team to avoid easily avoidable mistakes like working on the same files/issues etc. \\ \hline
        Developer & 
        \textbf{Readability:} When there are several developers, it is important that everyone writes code that is easy to read for others, as time flies quickly when you're trying to read bad code.
        \newline \textbf{Modifiability:} Adding functions and extensions to the game should be as easy as possible.
        \newline \textbf{Testability:} The game should be easy to test, to avoid bugs and errors after release. 
        \newline \textbf{Interoperability:} Since we have a server/client architecture, it is important that the two speaks well with each other. 
        \newline \textbf{Maintainability:} The program should be easy to maintain as bugs are found and updates has to be added.
        \newline \textbf{Availability:} Again, because of our chosen architecture, availability is a concern as the game should be available to play as much of the time as possible (the server should not be down unnecessarily). \\ \hline
        ATAM evaluator &
        \textbf{Reviewability:} The requirements have to be well-defined, and the documentation on these and on the architecture of the game has to be well-written.\\ \hline
        Course staff & 
        \textbf{Reviewability:} The code must be readable and "clean" to make it easy to review.
        \newline \textbf{Testability:} The code must be easy and comprehensible to run and test to make the testing phase a good experience.\\ \hline
\end{longtable}
\end{center}

